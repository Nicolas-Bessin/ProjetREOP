%Classe du document, A4, police taille 12
\documentclass[a4paper,12pt]{article}

% Dictionnaire français, pour caractères spéciaux, tirets, caractères accentués
\usepackage[french]{babel}
\usepackage[utf8]{inputenc}
%Toujours plus d'accents
\usepackage[T1]{fontenc}

\usepackage{indentfirst}
%Pour créer des paragraphes random
\usepackage{lipsum}  
%bibliographie
%Le style dépend du projet, voir avec le grand chef
%A mettre à l'endroit où vous voulez la faire apparaitre
%Donc dans le code pas ici t'as vu
%\bibliographystyle{ieeetr}
%\bibligraphy{nom_du_fichier.bib}

%hauteur entre deux lignes
\baselineskip 200cm
%hauteur entre deux paragraphes
\parskip 2mm
%longueur d'indentation
\parindent 2mm
%(on utilise \indent et \noindent sinon)

%Gérer ses marges

%Facilement
\usepackage[margin=2cm]{geometry}

%Précisément
%\usepackage[left=2cm , right=2cm, bottom=2cm, top=2cm, headheight=2cm]{geometry} 
%header c'est l'en-tête pas la marge supérieure

%Toujours plus précisément
%\addtolength{\oddsidemargin}{-0.5in}
%\addtolength{\evensidemargin}{-5cm}
%\addtolength{\topmargin}{-0.5in}

%Faires des articles  plusieures colonnes
\usepackage{multicol}
%Separation des colonnes
\setlength{\columnsep}{2cm}

%Avoir des entêtes et pieds de page stylés
\usepackage{fancyhdr}
\pagestyle{fancy}
%Pour enlever l'entête avec les sections
%\fancyhf{}

%Ca se fait sous format \<pos><type>{<contenu>}
%type c'est "head" ou "foot"
%pos pour position gauche "l", droite "r" ou centre "c"
%contenu c'est ce que tu mets dans dedans 
%marche aussi avec des images mais flemme
%mettre un trait
%\renewcommand{\footrulewidth}{1.5pt}

% Liens dans le document
\usepackage{hyperref}  
% Légendes dans les environnements "figure" et "float"
\usepackage{subcaption}
%La base pour faire des figures juste
\usepackage{graphicx}
\usepackage[export]{adjustbox}
\usepackage{wrapfig}
%Trucs utiles pour les maths
\usepackage{amsmath}

\begin{document}
\begin{titlepage}
    \begin{center}
        \vspace*{0.5cm}
        \includegraphics[scale=0.1]{logo_ponts.jpg}\\
        \vspace{0.7cm}
        {\Large ÉCOLE NATIONALE DES PONTS ET CHAUSSÉES}\\
        \vspace{4cm}
        \rule\linewidth{0.05cm}
        {\huge Recherche Opérationelle\par}
        \rule\linewidth{0.05cm}
        \vspace{1cm}
        {\Large Projet \par}
        \vspace{0.8cm}
        %{\Large }\\
        \vspace{0.3cm}
        {\large \textit{Nicolas Bessin}}\\
        \vspace{0.3cm}
        {\large \textit{Erwann Esteve}}\\
        \vspace{0.3cm}
        {\large \textit{Tidiane Polo}}\\
        \vspace{1.2cm}
        
    \end{center}
\end{titlepage}

\section{Questions}
On veut linéariser des contraintes:
\\
\begin {enumerate}
\item { 
    Contrainte de la forme : $\mu = \alpha \beta$ (1) avec $\alpha \in \lbrace 0,1 \rbrace$ et $\beta \in \lbrack 0, M \rbrack$ \\ 
    On a  $$(1) \iff 0 \leq \mu \leq \alpha M \text{  et  } \beta - (1 - \alpha)M \leq \mu \leq \beta$$
    }
\\
\item {
    Contrainte de la forme : $ \mu = \lbrack \beta \rbrack ^+$ avec $\beta \in \lbrack -M, M \rbrack$ \\
    On introduit une variable binaire $\alpha$ qui vaut 1 si $\beta > 0$ et 0 sinon. \\
    Pour cela, on introduit les contraintes suivantes :
    $$ 1 + \frac{\beta}{M} \geq \alpha \geq \frac{\beta}{M}$$
    La contrainte initiale s'écrit alors :
    $$ \mu = \alpha \beta $$
    et on peut la linéariser comme dans la question précédente. \\
    Cas limite : $\beta = 0$, on a alors $1 \geq \alpha \geq 0$. \\
    Ce n'est pas un problème puisque $\mu = \alpha \beta = 0$ quel que soit la valeur prise par $\alpha$
    }
\\
\item {
    Contrainte de la forme : $ \gamma = min(\alpha, \beta)$ avec $\alpha, \beta \in \lbrack -M, M \rbrack$ \\
    On introduit une variable binaire $\delta$ qui vaut 1 si $\beta < \alpha$ et 0 sinon. \\
    Pour cela, on introduit les contraintes suivantes :
    $$ \frac{\alpha - \beta}{2M} \leq \delta \leq 1 + \frac{\alpha - \beta}{2M}$$
    La contrainte initiale s'écrit alors :
    $$ \gamma = \delta \beta + (1 - \delta) \alpha $$
    et on peut la linéariser comme dans la question 1. \\
}
\\
\item {
    On a le problème suivant : 
    \begin{equation}
        \begin{aligned}
            \min _{\alpha, \beta, \gamma} & \max (\alpha, \beta) + \gamma \\
            \text{ s.c. } & A(\alpha, \beta, \gamma)^T \leq b \\
        \end{aligned}
    \end{equation}
    On introduit la variable continue $\delta$ et les contraintes suivantes :
    $$ \delta \geq \alpha $$
    $$ \delta \geq \beta $$
    Le problème initial s'écrit alors :
    \begin{equation}
        \begin{aligned}
            \min _{\alpha, \beta, \gamma, \delta}  \delta &+ \gamma \\
            \text{ s.c. }  A(\alpha, \beta, \gamma)^T &\leq b \\
            \alpha &\leq \delta \\
            \beta &\leq \delta \\
        \end{aligned}
    \end{equation}
}



\end{enumerate}
\end{document}

